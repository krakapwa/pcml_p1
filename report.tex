% Created 2016-10-16 Sun 13:01
\documentclass[11pt]{article}
\usepackage[utf8]{inputenc}
\usepackage[T1]{fontenc}
\usepackage{fixltx2e}
\usepackage{graphicx}
\usepackage{grffile}
\usepackage{longtable}
\usepackage{wrapfig}
\usepackage{rotating}
\usepackage[normalem]{ulem}
\usepackage{amsmath}
\usepackage{textcomp}
\usepackage{amssymb}
\usepackage{capt-of}
\usepackage{hyperref}
\usepackage{bm}
\usepackage{svg}
\usepackage{graphicx}
\graphicspath{{pics/}}
\usepackage[margin=1in]{geometry}
\author{Laurent Lejeune, Tatiana Fountoukidou, Guillaume de Montauzon}
\date{\today}
\title{Higgs Boson classification}
\hypersetup{
 pdfauthor={Laurent Lejeune, Tatiana Fountoukidou, Guillaume de Montauzon},
 pdftitle={Higgs Boson classification},
 pdfkeywords={},
 pdfsubject={},
 pdfcreator={Emacs 25.1.1 (Org mode 8.3.6)}, 
 pdflang={English}}
\begin{document}

\maketitle


\section{Data pre-processing}
\label{sec:orgheadline3}

\subsection{Data clean-up}
\label{sec:orgheadline1}
About 70\% of the samples contain missing values. Replacing missing values by the expectation over the valid samples leads to a distortion of the variable's distribution, which introduces a severe bias in regression procedures. A variant of K-Nearest-Neighbors strategy was implemented to fill-in missing values \cite{malarvizhi12}. To alleviate the computational cost, a random uniform sampling of the valid samples (samples without missing values) was performed prior to the nearest neighbors search.

\subsection{Data preparation}
\label{sec:orgheadline2}
Computing the eigen-values of the covariance matrix reveals that the second highest eigen value is xxx\% smaller than the highest.

\section{Method}
\label{sec:orgheadline5}
\subsection{Additive Logistic Regression \cite{friedman98}}
\label{sec:orgheadline4}


\bibliographystyle{ieeetr}
\bibliography{refs}
\printbibliography
\end{document}